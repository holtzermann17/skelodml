\documentclass{article}
\usepackage{fullpage}
\usepackage{latexml}
\usepackage{amsmath,amssymb,amsthm}
\usepackage{graphicx}
\usepackage{natbib}
\usepackage{hyperref}

\renewcommand{\P}{\mathbb{P}}
\newcommand{\E}{\mathbb{E}}
\DeclareMathOperator{\var}{var}
\DeclareMathOperator{\cov}{cov}
\newcommand{\deq}{\overset{\scriptscriptstyle{d}}{=}}

\newcommand{\given}{\,\vert\,}
\newcommand{\st}{\,\colon\,} % such that
\newcommand{\floor}[1]{{\left\lfloor #1 \right\rfloor }}
\newcommand{\one}{\mathbb{1}}

\newtheorem{definition}{Definition}
\newtheorem{lemma}{Lemma}
\newtheorem{theorem}{Theorem}
\newtheorem{exercise}{Exercise}

\catcode`\_=11\relax
\newcommand\email[1]{\_email #1\q_nil}
\def\_email#1@#2\q_nil{%
  \href{mailto:#1@#2}
    {{\emailfont\detokenize{#1}\emailampersat\detokenize{#2}}}%
}
\newcommand\emailfont{\rmfamily}
\newcommand\emailampersat{{\small@}}
\catcode`\_=8\relax 

\begin{document}

\title{ODLM focal areas}

\maketitle

\begin{enumerate}
\renewcommand{\theenumi}{\Alph{enumi}}
\item\label{ontology} an ontology for mathematics (or, for a start, a mathematical wordnet) (\href{https://github.com/holtzermann17/skelodml/wiki/A.-An-ontology-for-mathematics}{discuss})
\item\label{encoding} a semantically meaningful encoding of mathematical definitions, theorems, formulas, lemmas, \&c. (\href{https://github.com/holtzermann17/skelodml/wiki/B.-Semantically-meaningful-encodings-of-mathematical-objects}{discuss})
\item\label{search} semantic mathematical search (\href{https://github.com/holtzermann17/skelodml/wiki/C.-Semantic-mathematical-search}{discuss})
\item\label{fingerprinting} \href{http://www.ams.org/notices/201308/rnoti-p1034.pdf}{fingerprinting of mathematical theorems} (\href{https://github.com/holtzermann17/skelodml/wiki/D.-Fingerprinting-of-mathematical-theorems}{discuss})
\item\label{proof} tools for assisting mathematical proof based on semantically encoded mathematics (\href{https://github.com/holtzermann17/skelodml/wiki/E.-Tools-for-assisting-mathematical-proof}{discuss})
\item\label{tagging} manual and computerized (machine-learning based) tagging of mathematical papers (\href{https://github.com/holtzermann17/skelodml/wiki/F.-(semi)automated-tagging-of-mathematical-papers}{discuss})
\item\label{annotation} annotation tools for PDF files and scanned papers (\href{https://github.com/holtzermann17/skelodml/wiki/G.-annotation-tools-for-PDF-files}{discuss})
\item\label{authoring} support for open semantic docs and authoring workflows: (X)HTML, Kohlhase's stex, etc. (\href{https://github.com/holtzermann17/skelodml/wiki/H.-Open-semantic-docs-and-authoring-workflows}{discuss})
\item\label{community} community/authority management like mathoverflow, for management and production of collections (\href{https://github.com/holtzermann17/skelodml/wiki/I.-community-and-authority-management-tools}{discuss})
\item\label{data} structured data management tools: integration with Zotero, BibServer and other tools (\href{https://github.com/holtzermann17/skelodml/wiki/J.-structured-data-management-tools}{discuss})
\item\label{ocr} OCR for mathematics (\href{https://github.com/holtzermann17/skelodml/wiki/K.-OCR-for-mathematics}{discuss})
\end{enumerate}

We propose to offer:

\begin{enumerate}
\item\label{integration} a light-weight integration and organizational layer for coordinating work on the topics listed above (\href{https://github.com/holtzermann17/skelodml/wiki/1.-Home}{discuss})
\item\label{mail} an index of mailing lists relevant to the effort (\href{https://github.com/holtzermann17/skelodml/wiki/2.-Mailing-lists}{discuss})
\item\label{technology} An overview of relevant technologies (\href{https://github.com/holtzermann17/skelodml/wiki/3.-Technology-overview}{discuss})
\item\label{content} An overview of relevant sources of content (\href{https://github.com/holtzermann17/skelodml/wiki/4.-Content-overview}{discuss})
\end{enumerate}

\section{Background}

This site is being created as a response to the  \href{http://www.mathunion.org/fileadmin/CEIC/planning_wdml.pdf}{forthcoming article} and
 \href{http://arxiv.org/abs/1404.1905}{recent report} from Jim Pitman and collaborators about an Open Digital Mathematics Library (ODML).
%
The site is written with skelml, a simple tool that has recently been developed to use Github together with \LaTeX ML.  This model provides a nice, neutral, supported, and standards-compliant place to collaborate.  To join, please visit, and fork, \url{https://github.com/holtzermann17/skelodml} and send us pull requests.  You can also edit the \href{https://github.com/holtzermann17/skelodml/wiki}{wiki} or
submit \href{https://github.com/holtzermann17/skelodml/issues}{issues} directly.

\section{Plans}

We will try to describe the state of the art for items \ref{ontology}--\ref{ocr}, and check whether this list of focal areas is complete (it probably is not).  Once we know that we have a reasonable outline, we can look at the division of work.  Often, small subsets of projects may fit together into a grant proposal.  For example, items \ref{annotation}, \ref{authoring} and \ref{ocr} might fit naturally together, if we want to have annotated versions of legacy document that correspond on a line-by-line basis to open docs on the web.

\section{Example}

To illustrate the sort of thing that can be usefully stored in this repository, we've added a document with \href{http://holtzermann17.github.io/skelodml/bibserver-setup.html}{instructions} for installing BibServer, one of the relevant technologies.  Eventually, this repository will contain many similar how-tos and useful pointers to tools and content.

\end{document}
