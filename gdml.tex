\documentclass{article}
\usepackage{fullpage}
\usepackage{latexml}
\usepackage{amsmath,amssymb,amsthm}
\usepackage{graphicx}
\usepackage{natbib}
\usepackage{hyperref}

\renewcommand{\P}{\mathbb{P}}
\newcommand{\E}{\mathbb{E}}
\DeclareMathOperator{\var}{var}
\DeclareMathOperator{\cov}{cov}
\newcommand{\deq}{\overset{\scriptscriptstyle{d}}{=}}

\newcommand{\given}{\,\vert\,}
\newcommand{\st}{\,\colon\,} % such that
\newcommand{\floor}[1]{{\left\lfloor #1 \right\rfloor }}
\newcommand{\one}{\mathbb{1}}

\newtheorem{definition}{Definition}
\newtheorem{lemma}{Lemma}
\newtheorem{theorem}{Theorem}
\newtheorem{exercise}{Exercise}

\catcode`\_=11\relax
\newcommand\email[1]{\_email #1\q_nil}
\def\_email#1@#2\q_nil{%
  \href{mailto:#1@#2}
    {{\emailfont\detokenize{#1}\emailampersat\detokenize{#2}}}%
}
\newcommand\emailfont{\rmfamily}
\newcommand\emailampersat{{\small@}}
\catcode`\_=8\relax 

\begin{document}

\title{Global Digital Mathematics Library}

\maketitle

\section{Background (adapted from a summary by Patrick Ion)}

Following the IMU meeting on August 17 at which a GDML WG was named,
as a result of the enthusiasm for the matter that people in Seoul
displayed, we had an informal meeting that has come to be known as the
``Fried Chicken" meeting. 

Those participating included WG members and some others:
\begin{itemize}
\item Thierry Bouche [WG]
\item Tom Hales
\item Patrick Ion [WG]
\item Michael Kohlhase [WG]
\item Alex Pechen
\item Olaf Teschke [WG]
\item Michael Trott
\item Eric Weisstein [WG]
\item Stephen Wolfram
\end{itemize}

The discussion was largely led by Stephen Wolfram with Michael
Kohlhase recording matters on the whiteboard.  It ranged over material
familiar in some respects to all of us, but the discussions helped the
group realize better their common purpose.  Obviously only part of the
context could be broached, and we already have the much larger NRC
report to work with.

Rough notes derived from the board are as follows, organized as a
number of Goal points, as follows:

\begin{itemize}
\item[{[}G1{]}] \textbf{Material Acquisition - Digitize All Mathematics.}
The general idea is to digitize all mathematical documents so as to
have the legacy of the mathematical corpus available. Although
clearly unrealizable, it is a good simple goal.

\item[{[}G2{]}] \textbf{IDs and Metadata \(\forall\) Math.}
Indexing must be provided and  databases produced about the
documents of mathematics that are assembled.

\item[{[}G3{]}] \textbf{Add Presentation Markup \(\forall\) Math (pMathML /\LaTeX).}
In particular, \(\TeX\), character PDF, and markups that specify
the documents in a way that is more computationally accessible
should be provided.

\item[{[}G4{]}] \textbf{Create a Target Format for Semantic Math.}
An intermediate-level semantic representation language for
mathematical literature, between presentation and full formalization,
and applicable to enhance the availability if the knowledge in the
mathematical corpus needs developing.

\item[{[}G5{]}] \textbf{Advanced Services and Use Cases beyond {[}G2-G4{]}.}

\begin{itemize}
\item "Wolfram Alpha" for all math literature
\item Conflict propagation over the literature (we need to know what is true)
\item Topic modeling for mathematics
\item Support for such queries as "What papers say something about my paper?"
\item Finding results that are implicit in a document set
\item Finding  theorems applicable in a context
\item Illuminating tenure and promotion cases
\item Facilitating search for plagiarism and reinvention
\item Mathematical recommender systems
\end{itemize}

\item[{[}G6{]}] \textbf{WDML Portal(s)/App(s)/API(s).}
A GDML will offer new services through Apps and must offer
public APIs to its data.

\item[{[}G7{]}] \textbf{Models of Sustainability and Extensions.}
The organizational model of a GDML for the ages has to be set up.

\item[{[}G8{]}] \textbf{Subcontract Processing for Funds / Hiring for WDML.}
In the initial stages there has to be an organizational center
that can receive funds, enter into contracts and hire workers.

\item[{[}G9{]}] \textbf{Mathematical linguistics.}
The language of mathematics needs more careful analysis extending
natural language processing.
\end{itemize}

\section{Additional matters not listed as G points.}

\begin{itemize}
\item GDML needs lawyerly advice in setting it up.
\item Should one build a Math Papers App Store?
\item A GDML should support data-mining behind an NDA wall if needed.
\item Holdings of a GDML must be physically distributed and redundant.
  LOCKSS!
\item Museum of Math?
\item GDML will have to engage in outreach throughout its existence.
\end{itemize}


\end{document}
